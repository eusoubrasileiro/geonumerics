\documentclass[legalpaper, 12pt]{article}
\usepackage[utf8]{inputenc}
\usepackage[english]{babel}
\usepackage{amsmath}
\usepackage{amsfonts}
\usepackage{amssymb}
\usepackage{xcolor}

\begin{document}

\textbf{Explicit 2D acoustic wave equation }

The scalar wave equation, a fluid medium with variable velocity, is defined by (\ref{1}) Where $U$ and $V$ represents the pressure (or scalar displacement) and velocity fields and S(t) the source, and can be simplified in (\ref{2}) for 2D case:

\begin{eqnarray}
\frac{\partial^2 U}{\partial t^2} \frac{1}{V^2} &=& \nabla^2 U + S(t) \label{1} \\
\frac{\partial^2 U}{\partial t^2} &=& V^2 \left( \frac{\partial^2 U}{\partial x^2}+ \frac{\partial^2 U}{\partial z^2} +S(t) \right)
\label{2}
\end{eqnarray}

From equation (\ref{1}) we can approximate derivatives by Taylor series, manipulating we can reach backward differences in time and centered in space first and second order; using $Dx=Dz=Ds$, remember that $j=jDs$, $k=kDs$ and $n=nDt$.

\begin{multline}
U_{jk}^{n+1}  =  \left( \frac{\Delta t  V_{jk} }{\Delta s} \right) ^2 \left(  \sum_{a=-N}^N w_a U_{j+a k}^n + \sum_{a=-N}^N w_a U_{j k+a}^n +S_{jk}^n {\Delta s}^2 \right) + 2 U_{jk}^{n} - U_{jk}^{n-1}  \\
U_{jk}^{n+1}  =  \left( \frac{\Delta t  V_{jk}}{\Delta s} \right) ^2   \left[ \sum_{a=-N}^N  w_a \left( U_{j+a k}^n + U_{j k+a}^n \right) \right] + S_{jk}^n {\Delta t}^2 V_{jk}^2 + 2 U_{jk}^{n} - U_{jk}^{n-1} \\
U_{jk}^{n+1}  =  \left( \frac{\Delta t  V_{jk}}{\Delta s} \right) ^2  \left[ \sum_{a=-N}^N  w_a \left( U_{j+a k}^n + U_{j k+a}^n \right) + S_{jk}^n {\Delta s}^2\right] + 2 U_{jk}^{n} - U_{jk}^{n-1} \label{3}
\end{multline}

For forth order space, we have $N=2$ and $w$ is:
$$ w = \frac{1}{12} [-1, 16, -30, 16, -1] $$

Where, $ R $ must allways be bounded by the stability criteria.

\[ R = \frac{\Delta t  V_{max}}{\Delta s} \]

Where $V_{max}$ is the maximum velocity.

Simplified from Chen [2] the forth order schema requires :

$$ \Delta t \leq \frac{2 \Delta s}{ V_{max} \sqrt{2/12(1+16+30+16+1)}} \implies \Delta t \leq \frac{ \Delta s \sqrt{3}}{ V_{max} \sqrt{8}} = \frac{ \Delta s \sqrt{3}}{ V_{max} \sqrt{8}}$$

Notes regarding implementation:

\begin{itemize}
\item Numerical aproximations turn the $R = \frac{\Delta t  V_{max}}{\Delta s}$ in higher than accepted deppending on the way the sums and products are done
\item The Laplacian should not be contaminated, should just multiply by R.
\item The last two lines in the equation inclunding equation (\ref{3}) are acceptable implementations.
\end{itemize}

References

[1] Alford R.M., Kelly K.R., Boore D.M. (1974) Accuracy of finite-difference modeling of the acoustic wave equation Geophysics, 39 (6), P. 834-842

[2] Chen, Jing-Bo (2011) A stability formula for Lax-Wendroff methods with fourth-order in time and general-order in space for the scalar wave equation Geophysics, v. 76, p. T37-T42
 

\end{document}