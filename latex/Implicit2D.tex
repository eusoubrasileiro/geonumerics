\documentclass[legalpaper, 12pt]{article}
\usepackage[utf8]{inputenc}
\usepackage[english]{babel}
\usepackage{amsmath}
\usepackage{amsfonts}
\usepackage{amssymb}
\usepackage{xcolor}

\begin{document}

\textbf{Implicit 2D acoustic wave equation }

The scalar wave equation, a fluid medium with variable velocity, is defined by:

\[ \frac{\partial^2 U}{\partial t^2} = V^2 \nabla^2 U \]

Where $U$ and $V$ represents the pressure (displacement) and velocity fields. For 2D case it gets simplified to:

\begin{equation}
\frac{\partial^2 U}{\partial t^2} = V^2 \left( \frac{\partial^2 U}{\partial x^2}+ \frac{\partial^2 U}{\partial z^2} \right)
\label{1}
\end{equation}

From equation (\ref{1}) we can approximate derivatives by Taylor series, manipulating we can reach backward differences in time and centered in space first and second order; using $Dx=Dz=Ds$, remember that $j=jDs$, $k=kDs$ and $n=nDt$.

\begin{eqnarray}
\frac{U_{jk}^n -2 U_{jk}^{n-1} + U_{jk}^{n-2}}{\Delta t^2} =  \left( \frac{V_{jk}^n}{\Delta s} \right) ^2 \left(  U_{j+1k}^n - 4 U_{jk}^n + U_{jk+1}^n + U_{j-1k}^n + U_{jk-1}^n  \right)
\label{2} \\
U_{jk}^n  =  \left( \frac{\Delta t  V_{jk}^n}{\Delta s} \right) ^2 \left(  U_{j+1k}^n - 4 U_{jk}^n + U_{jk+1}^n + U_{j-1k}^n + U_{jk-1}^n  \right) + 2 U_{jk}^{n-1} - U_{jk}^{n-2} \nonumber
\end{eqnarray}

Rearranging to solve in matrix form for the unknowns $ U^n $ (time n) around $U_{jk}$, we get 

\begin{multline}
 U_{jk}^{n-2} -2 U_{jk}^{n-1} =  \left( \frac{\Delta t  V_{jk}^n}{\Delta s} \right) ^2 \left[  U_{j+1k}^n - 4 U_{jk}^n - \left( \frac{\Delta s}{\Delta t  V_{jk}^n} \right) ^2  U_{jk}^n + U_{jk+1}^n + U_{j-1k}^n + U_{jk-1}^n  \right] \\
\left( \frac{\Delta s}{ \Delta t  V_{jk}^n} \right) ^2 \left( U_{jk}^{n-2} -2 U_{jk}^{n-1} \right) = U_{j+1k}^n - U_{jk}^n \left[ 4+ \left( \frac{\Delta s}{\Delta t  V_{jk}^n} \right) ^2  \right]  + U_{jk+1}^n + U_{j-1k}^n + U_{jk-1}^n \label{3}
\end{multline}

\textit{Note:} Centered differences in  (\ref{2}) would reach a explicit method with stability restrictions. So although (\ref{2}) is first order in time the analyses in the end shows it's unconditionally stable.

\newpage

Rewritting (\ref{3}) to a simplified form, introducing some new variables: 

\begin{multline}
\left( \frac{\Delta s}{ \Delta t  V_{jk}^n} \right) ^2 \left( U_{jk}^{n-2} -2 U_{jk}^{n-1} \right) =
  U_{j-1k}^n - \left[ 4+ \left( \frac{\Delta s}{\Delta t  V_{jk}^n} \right) ^2  \right] U_{jk}^n + U_{j+1k}^n + U_{jk+1}^n + U_{jk-1}^n  \\
r^{-2} \left( U_{jk}^{n-2} -2 U_{jk}^{n-1} \right) =
  U_{j-1k}^n + \gamma \  U_{jk}^n + U_{j+1k}^n + U_{jk+1}^n + U_{jk-1}^n 
\label{4}
\end{multline}

Where: 

$$ r = \frac{\Delta t  V_{jk}^n}{ \Delta s} $$
$$ \gamma = -4 -r^{-2}  $$

\

For a tiny grid 3x3 dimensions we can apply a Dirichlet boundary condition in (\ref{4}). That means $ U_{\Omega} = 0 \ \forall \ n $ and $ U_{jk}^n $ with $ j, \ k \in Z \ \left[ 0, 2  \right] $ , ilustrated bellow: \\ 
\newline

\begin{tabular}{p{7cm}p{7cm}}
{ \begin{tabular}{| c | c | c | c | c |}
	\hline
	$U_{\Omega}$ & $U_{\Omega}$ & $U_{\Omega}$ & $U_{\Omega}$ & $U_{\Omega}$ \\ \hline
	$U_{\Omega}$ & { \color{red} $U_{00}^n$}& { \color{red} $U_{10}^n$ } & { \color{red} $U_{20}^n$ } & $U_{\Omega}$ \\ \hline
	$U_{\Omega}$ & { \color{red} $U_{10}^n$ } & { \color{red} $U_{11}^n$ } & { \color{red} $U_{21}^n$ } & $U_{\Omega}$\\ \hline
	$U_{\Omega}$ & { \color{red} $U_{20}^n$ } & { \color{red} $U_{21}^n$ } & { \color{red} $U_{22}^n$ } & $U_{\Omega}$\\ \hline
	$U_{\Omega}$ & $U_{\Omega}$ & $U_{\Omega}$ & $U_{\Omega}$ & $U_{\Omega}$ \\ \hline
\end{tabular} }
(4.a)
&
{ \begin{tabular}{| c | c | c | c | c |}
	\hline
	0.0 & 0.0 & 0.0 & 0.0 & 0.0 \\ \hline
	0.0 & { \color{red} $P_{0}$ } & { \color{red} $P_{1}$ } & { \color{red} $P_{2}$ } & 0.0 \\ \hline
	0.0 & { \color{red} $P_{3}$ } & { \color{red} $P_{4}$ } & { \color{red} $P_{5}$ } & 0.0 \\ \hline
	0.0 & { \color{red} $P_{6}$ } & { \color{red} $P_{7}$ } & { \color{red} $P_{8}$ } & 0.0 \\ \hline
	0.0 & 0.0 & 0.0 & 0.0 & 0.0 \\ \hline
\end{tabular} }
(4.b)
\end{tabular}


\
\newline

We can assembly the linear system for this simple grid as a 9x9 matrix, using the equation (\ref{4}) for every unknown $ P_i $ . Not showing the independent term $  r^{-2} \left( U_{jk}^{n-2} -2 U_{jk}^{n-1} \right) $. Calculation of the coeficients can be made simple  using (4.b), empty position means 0 as coeficiente; also represented by $ \varnothing $.

\begin{center}
\begin{tabular}{ l }
$ P_0 = \varnothing  0  1  \varnothing 3  $ \\
$ P_1 = 0 1 2 \varnothing 4 $ \\
$ P_2 = 1 2 \varnothing \varnothing 5 $ \\
$ P_3 = \varnothing 3 4 0 6 $\\
$ P_4 = 3 4 5 1 7 $\\
$ P_5 = 4 5 \varnothing 2 8 $\\
$ P_6 = \varnothing 6 7 3 \varnothing $\\
$ P_7 = 6 7 8 4 \varnothing $\\
$ P_8 = 7 8 \varnothing 5 \varnothing $\\
\end{tabular}
\end{center}

\begin{center}
\begin{tabular}{| c | c | c | c | c | c | c | c | c | c |}
\hline
	        & $ P_0 $ & $ P_1 $ & $ P_2 $ & $ P_3 $ & $ P_4 $ & $ P_5 $ & $ P_6 $ & $ P_7 $ & $ P_8 $ \\ \hline
	$ P_0 $ &$\gamma$ &    1    &         &    1    &         &         &         &         &         \\ \hline
	$ P_1 $ &    1    &$\gamma$ &     1   &         &     1   &         &         &         &         \\ \hline
	$ P_2 $ &         &    1    &$\gamma$ &         &         &    1    &         &         &         \\ \hline
	$ P_3 $ &    1    &         &         &$\gamma$ &    1    &         &    1    &         &         \\ \hline
	$ P_4 $ &         &    1    &         &    1    &$\gamma$ &    1    &         &    1    &         \\ \hline
	$ P_5 $ &         &         &    1    &         &    1    &$\gamma$ &         &         &    1    \\ \hline
	$ P_6 $ &         &         &         &    1    &         &         &$\gamma$ &    1    &         \\ \hline
	$ P_7 $ &         &         &         &         &    1    &         &    1    &$\gamma$ &    1    \\ \hline
    $ P_8 $ &         &         &         &         &         &    1    &         &    1    &$\gamma$ \\ \hline
\end{tabular} 
\end{center}


Or the full human form, matrix notation using the independent term :

\begin{eqnarray}
\begin{pmatrix}
\gamma   &    1    &         &    1    &         &         &         &         &         \\ 
    1    & \gamma  &     1   &         &     1   &         &         &         &         \\ 
         &    1    & \gamma  &         &         &    1    &         &         &         \\ 
    1    &         &         & \gamma  &    1    &         &    1    &         &         \\ 
         &    1    &         &    1    & \gamma  &    1    &         &    1    &         \\ 
         &         &    1    &         &    1    & \gamma  &         &         &    1    \\ 
         &         &         &    1    &         &         & \gamma  &    1    &         \\ 
         &         &         &         &    1    &         &    1    & \gamma  &    1    \\ 
         &         &         &         &         &    1    &         &    1    & \gamma  \\
\end{pmatrix} 
\begin{pmatrix}
 \ U_{00}^n \ \\
 \ U_{01}^n \ \\ 
 \ U_{02}^n \ \\ 
 \ U_{10}^n \ \\ 
 \ U_{11}^n \ \\ 
 \ U_{12}^n \ \\ 
 \ U_{20}^n \ \\ 
 \ U_{21}^n \ \\ 
 \ U_{22}^n \ \\ 
\end{pmatrix}
=
r^{-2} 
\begin{pmatrix}
U_{00}^{n-2} \\
U_{01}^{n-2} \\ 
U_{02}^{n-2} \\ 
U_{10}^{n-2} \\ 
U_{11}^{n-2} \\ 
U_{12}^{n-2} \\ 
U_{20}^{n-2} \\ 
U_{21}^{n-2} \\ 
U_{22}^{n-2} \\
\end{pmatrix}
-2r^{-2} 
\begin{pmatrix}
U_{00}^{n-1} \\
U_{01}^{n-1} \\ 
U_{02}^{n-1} \\ 
U_{10}^{n-1} \\ 
U_{11}^{n-1} \\ 
U_{12}^{n-1} \\ 
U_{20}^{n-1} \\ 
U_{21}^{n-1} \\ 
U_{22}^{n-1} \\
\end{pmatrix}
\end{eqnarray}



Analysing above a recursive schema can be created. From a grid of dimensions $ N_x \times N_z $ we have a linear system $L$ with dimension $ M \times M $ ( $ M = N_x N_z $) . We have five rules for every line in $ L $, looking at $ L $ as a list of lines, list with dimension $ M $, these rules are based upon the principal diagonal, defined as $L[i][i]$.
\newline
\ 
Where $ i = \ \in [0, M] $

\begin{eqnarray}
	j &=& i \% N_x \nonumber \ \  \mbox{integer divison remainder} \\ 
	k &=& i / N_x \nonumber \ \ \mbox{integer division} \\
	U_{j-1k}^n &=& 1 \ \ \forall \ j-1 \geq 0 \implies L[i][i-1] = 1 \nonumber \\
	U_{jk}^n &=&  \gamma \ \ \implies L[i][i] = \gamma \nonumber \\
	U_{j+1k}^n &=& 1 \ \ \forall \ j+1 < N_x \implies L[i][i+1] = 1 \nonumber \\
	U_{jk-1}^n &=& 1 \ \ \forall \ z-1 \geq 0 \implies L[i][i-N_x] = 1 \nonumber \\
	U_{jk+1}^n &=& 1 \ \ \forall \ z+1 < N_z \implies L[i][i+N_x] = 1 \nonumber 
\end{eqnarray}

Can also be seeing, looking each diagonal as a list (reminded that we have 5 defined diagonals): 

* Leading diagonal defined by $L[p] = \gamma_{jk}$ with $p \in [0, M] $

* Leading diagonal upper +1 $L[p] = 1$ with $p \in [0, M-1] $ \\
   Leading diagonal lower -1  $L[p] = 1$ with $p \in [0, M-1] $ \\
   Except where $ p \%Nx == 0 $

* Leading diagonal upper in $+Nx$ $L[p] = 1$ with $p \in [0, M-Nx] $ \\
   Leading diagonal lower in $-Nx$  $L[p] = 1$ with $p \in [0, M-Nx] $ \\
   
This matrix similar to the 2d poisson matrix, with $5N_xN_z-2N_x-2N_z$ non zero elements. 


\newpage

\
\textbf{Von Newman stability analysis}

Von Newman stability analysis for acoustic wave equation implicit centered differences: 1st order time and space (N 2)'th order:


\begin{multline}
U_{jk}^{n+2}  =  \left( \frac{\Delta t  V_{jk} }{\Delta s} \right) ^2 \left(  \sum_{a=-N}^N w_a U_{j+a k}^{n+2} + \sum_{a=-N}^N w_a U_{j k+a}^{n+2} \right) + 2 U_{jk}^{n+1} - U_{jk}^n  \\
U_{jk}^{n+2}  =  \left( \frac{\Delta t  V_{jk}}{\Delta s} \right) ^2  \sum_{a=-N}^N  w_a \left( U_{j+a k}^{n+2} + U_{j k+a}^{n+2} \right) + 2 U_{jk}^{n+1} - U_{jk}^n \label{a6}
\end{multline}

For forth order space, we have $N=2$ and $w$ is:
$$ w = \frac{1}{12} [-1, 16, -30, 16, -1] $$

Can be simplified to 1st order (N=1):

\begin{equation}
U_{jk}^{n+2}  =  \left( \frac{\Delta t  V_{jk}}{\Delta s} \right) ^2 \left(  U_{j+1k}^{n+2} - 4 U_{jk}^{n+2} + U_{jk+1}^{n+2} + U_{j-1k}^{n+2} + U_{jk-1}^{n+2}  \right) + 2 U_{jk}^{n+1} - U_{jk}^{n} \nonumber
\end{equation}


Using the discrete solution for 2D wave equation, where $ i = \sqrt{-1} $, $ n = n \Delta t $, $ j = j \Delta x $ and $ k = k \Delta z $. Last using $ \Delta x = \Delta z = \Delta s $, follows that the discrete solution can be written as:

\begin{eqnarray}
U_{jk}^n = e^{i \left( \omega t + px + qz \right)} \nonumber \\
U_{jk}^n = \epsilon^n e^{i \left( pj\Delta s + qk\Delta s \right)}  \nonumber \\
U_{jk}^n = \epsilon^n e^{i \Delta s \left( pj + qk \right)}  \label{a7}
\end{eqnarray}

Where $\epsilon $ is the growth factor, and should be $ |\epsilon| \leq 1$ for stability. \\

Replacing (\ref{a7}) in (\ref{a6}), using the identities bellow and simplifying dividing both sides by $ U_{jk}^{n+2} $

$$ r = \frac{\Delta t  V_{jk}}{\Delta s} $$
$$ \phi_{j+l\ k+m} = e^{i \Delta s \left( pl+qm \right)} $$

\begin{equation}
\Omega = r^2 \sum_{a=-N}^N  w_a \left( \phi_{j+a k} + \phi_{j k+a}  \right) \label{a8}
\end{equation}

we get:

\begin{eqnarray}
1  =  \Omega + 2 \epsilon^{-1} -\epsilon^{-2} \nonumber \\
\quad \text{making} \ \ \epsilon^{-1} = \mu \nonumber \\
\mu^2 - 2 \mu + 1 -\Omega = 0 \nonumber \\
\mu = \frac{ 2 \pm \sqrt{ 4 - 4 +4 \Omega}}{2} \nonumber \\
\mu = 1 \pm \sqrt{ \Omega} \label{a9}
\end{eqnarray}

back to expand $ \Omega $ defined in (\ref{a8}):

\begin{eqnarray}
\Omega &=& r^2 \sum_{a=-N}^N  w_a \left( \phi_{j+a k} + \phi_{j k+a}  \right) \nonumber \\
	&=& r^2 \sum_{a=-N}^{N} w_a ( e^{i \Delta s \ p a} + e^{i \Delta s \ q a} )\nonumber 
\end{eqnarray}
\begin{equation}
=r^2
\begin{pmatrix}
 \cdots & e^{-i \Delta s 2 p} + e^{-i \Delta s 2 q} & e^{-i \Delta s p} + e^{-i \Delta s q} & e^0+e^0 & e^{i \Delta s p} + e^{i \Delta s q} & e^{i \Delta s 2 p} + e^{i \Delta s 2 q} & \cdots \\
\end{pmatrix}
\begin{pmatrix}
\cdots \\
w_{-2} \\
w_{-1} \\
w_0 \\
w_1 \\
w_2 \\
\cdots
\end{pmatrix}
\nonumber
\end{equation}

Since $w$ is even $ w_a = w_{-a} $ and $ e^{i\theta} + e^{-i\theta} = 2 \cos{\theta} $ we can rewrite as:

\begin{equation}
=r^2
\begin{pmatrix}
 \cdots & 2\cos( \Delta s 2 p) + 2\cos(\Delta s 2 q) & 2\cos(\Delta s p) + 2\cos(\Delta s q) & 2 \\
\end{pmatrix}
\begin{pmatrix}
\cdots \\
w_{2} \\
w_{1} \\
w_0 \\
\end{pmatrix}
\nonumber
\end{equation}

For the simplest case 2nd order $ N=1 $ we have $ (w_1, w_0) = (1, -2) $

\begin{eqnarray}
\Omega	&=& r^2 \left( 2\cos(\Delta s p) + 2\cos(\Delta s q) - 4\right) \nonumber \\
	&=& -4r^2 \left( \sin^2(\frac{\Delta s p}{2}) + \sin^2(\frac{\Delta s q}{2}) \right) \label{a10}
\end{eqnarray}

Note:  $ 2 \cos(\theta) - 2 = -4 \sin ^2 (\theta) $ .\\

Replacing back to (\ref{a9}) :

\begin{eqnarray}
\mu &=& 1 \pm \sqrt{\Omega} \nonumber \\
\mu &=& 1 \pm  i 2 \sqrt{r^2 \left( \sin^2(\frac{\Delta s p}{2}) + \sin^2(\frac{\Delta s q}{2}) \right)} \nonumber
\end{eqnarray}

Back to $\epsilon$ :

$$ \frac{1}{a^2+b^2} = | \epsilon^{'} | = | \epsilon^{''} | = 
\frac{1}{1 + 4r^2 \left( \sin^2(\frac{\Delta s p}{2}) + \sin^2(\frac{\Delta s q}{2}) \right)} $$
\\
\textit{That is for $ \forall \ r, \ \Delta \ s, \ p, \ q \ \implies | \epsilon | \leq 1 $ 
Follows (\ref{a6}) with $N=1$ is unconditionally stable. }

\end{document}